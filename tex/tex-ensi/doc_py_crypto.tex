\documentclass[a4paper, 12pt]{report}
\usepackage[utf8]{inputenc}
\usepackage{amsmath}
\usepackage[english, francais]{babel}
\usepackage{geometry}
\usepackage{enumitem}
\usepackage{graphicx}
\usepackage{hyperref}
\usepackage{listings}
\usepackage{color}
\usepackage{fancyhdr}
\pagestyle{fancy}
\usepackage{tikz}

\usetikzlibrary{arrows,shapes,positioning,shadows,trees}
\tikzset{
  basic/.style  = {draw, text width=3cm, drop shadow, font=\sffamily, rectangle},
  root/.style   = {basic, rounded corners=2pt, thin, align=center,
                   fill=blue!30},
  level 2/.style = {basic, rounded corners=6pt, thin,align=center, fill=blue!40,
                   text width=8em},
  level 3/.style = {basic, thin, align=left, fill=gray!40, text width=6.5em}
}

\definecolor{mygreen}{rgb}{0,0.6,0}
\definecolor{mygray}{rgb}{0.5,0.5,0.5}
\definecolor{mymauve}{rgb}{0.58,0,0.82}
\definecolor{background}{rgb}{0.97, 0.97, 0.97}



\lstset{ % Settings de la coloration de code
  backgroundcolor=\color{background},   % choose the background color; you must add \usepackage{color} or \usepackage{xcolor}
  basicstyle=\footnotesize,        % the size of the fonts that are used for the code
  breakatwhitespace=false,         % sets if automatic breaks should only happen at whitespace
  breaklines=true,                 % sets automatic line breaking
  captionpos=b,                    % sets the caption-position to bottom
  commentstyle=\color{mygreen},    % comment style
  deletekeywords={...},            % if you want to delete keywords from the given language
  escapeinside={\%*}{*)},          % if you want to add LaTeX within your code
  extendedchars=true,              % lets you use non-ASCII characters; for 8-bits encodings only, does not work with UTF-8
  frame=single,                    % adds a frame around the code
  keepspaces=true,                 % keeps spaces in text, useful for keeping indentation of code (possibly needs columns=flexible)
  keywordstyle=\color{blue}\textbf,       % keyword style
  language=Octave,                 % the language of the code
  morekeywords={*,...},            % if you want to add more keywords to the set
  numbers=none,                    % where to put the line-numbers; possible values are (none, left, right)
  numbersep=5pt,                   % how far the line-numbers are from the code
  numberstyle=\tiny\color{mygray}, % the style that is used for the line-numbers
  rulecolor=\color{black},         % if not set, the frame-color may be changed on line-breaks within not-black text (e.g. comments (green here))
  showspaces=false,                % show spaces everywhere adding particular underscores; it overrides 'showstringspaces'
  showstringspaces=false,          % underline spaces within strings only
  showtabs=false,                  % show tabs within strings adding particular underscores
  stepnumber=1,                    % the step between two line-numbers. If it's 1, each line will be numbered
  stringstyle=\color{mymauve},     % string literal style
  tabsize=2,                       % sets default tabsize to 2 spaces
  title=\lstname                   % show the filename of files included with \lstinputlisting; also try caption instead of title
}


\geometry{vmargin=2cm}
\setlength{\headheight}{16pt}

\renewcommand{\footrulewidth}{1pt}
\fancyfoot[C]{\textbf{page \thepage}}
\fancyhead[L]{}

\fancypagestyle{plain}{
  \fancyfoot[C]{\textbf{page \thepage}}
}

\title{\bf{Documentation de la librairie PyCrypto}}
\author{Laurent PARMENTIER\\ Quentin DERORY\\ Jean-Baptiste MAURANYAPIN \\ \\ Encadrant : Patrick LACHARME}
\date{04/10/2014}

\begin{document}

\maketitle

\tableofcontents

\chapter{Fonction de hachage}

\section{MD5}

L'algorithme MD5, est une fonction de Hash, qui était très utilisé dans le domaine web \footnote{\url{https://github.com/}}.

\begin{lstlisting}
#!/usr/bin/python 
from Crypto.Hash import MD5

hash = MD5.new()
hash.update("hello world")
print hash.hexdigest()
\end{lstlisting}

\section{SHA-256}

\begin{lstlisting}
#!/usr/bin/python
from Crypto.Hash import SHA256

hash = SHA256.new()
hash.update("hello world")
print hash.hexdigest() # print hash sha256 of "hello world"
\end{lstlisting}

\chapter{Chiffrement a cle prive}

\section{3DES}

\begin{lstlisting}
#!/usr/bin/python
from Crypto import Random
from Crypto.Cipher import DES3

key = b'une cle 16 octet'
iv = Random.new().read(DES3.block_size)
des3 = DES3.new(key, DES3.MODE_ECB, iv)
print des3.encrypt("hello wo") # obligatoire d'avoir une longueur de mot multiple de 8
#TODO mot sur dechiffrement
\end{lstlisting}

\section{AES}

\begin{lstlisting}
#!/usr/bin/python
from Crypto import Random
from Crypto.Cipher import AES

key = b'une cle 16 octet'
iv = Random.new().read(AES.block_size)
aes = AES.new(key, AES.MODE_ECB, iv)
print aes.encrypt("hello world     ") # obligatoire d'avoir une longueur de mot multiple de 16
#TODO mot sur dechiffrement
\end{lstlisting}

\section{ARC4}

\begin{lstlisting}
#!/usr/bin/python
from Crypto.Cipher import ARC4
key = b'Tres longue cle confidentielle'
arc4 = ARC4.new(key)
print arc4.encrypt("Un message    ")
\end{lstlisting}

\section{CAST}

\begin{lstlisting}

\end{lstlisting}

\section{BLOWFISH}

\end{document}